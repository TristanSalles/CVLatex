\documentclass[11pt,a4paper,sans]{moderncv}
% moderncv styles
%\moderncvstyle{casual}
\moderncvstyle{casual}     
%\moderncvstyle{classic}
%\moderncvstyle[roman]{classic}
\moderncvcolor{grey}                               
% character encoding
\usepackage[applemac]{inputenc}
\usepackage{fontawesome}

\setlength\arrayrulewidth{.4pt}
\setlength\tabcolsep{5pt}

\usepackage[scale=0.85]{geometry}

% personal data
\name{Tristan}{Salles}
\title{Software Development \& Data Science}
\address{Lecturer}{School of Geosciences --  The University of Sydney} {} 
\phone[none]{ Dr Tristan Salles}
%\email{tristan.salles@csiro.au}
\photo[30pt][0.0pt]{logo} 


%----------------------------------------------------------------------------------
%            content
%----------------------------------------------------------------------------------
\begin{document}
\makecvtitle

\social[github]{tristan-salles}
\vspace{-2.5em}
\section{Personal details}
\cvitem{Civility}{\small Dr Tristan Salles (Ph.D.) 
\hfill{}\break citizenship: French, Australian
\hfill{}\break date of birth: 10/10/1979} 

\cvitem{Languages}{French: native | English: fluent | German: basic}
\cvitem{Contact}{Rm 454, Madsen F09, The University of Sydney, NSW 2006, Australia
\hfill{}\break \fixedphonesymbol +61 2 8627 4123  \hspace{0.6em} \mobilephonesymbol +61 4 5146 2502 
\hfill{}\break \emaillink[\emailsymbol tristan.salles@sydney.edu.au]{tristan.salles@sydney.edu.au} 
\hspace{0.6em} \link[\githubsocialsymbol tristan-salles]{https://github.com/tristan-salles}
\hfill{}\break \hspace{0.6em} \link[\linkedinsocialsymbol tristan-salles]{http://www.linkedin.com/in/sallestristan}
\hspace{0.6em} \link[\twittersocialsymbol $@$salles\_tristan]{http://twitter.com/salles_tristan}}

%\vspace{1.5em}
%-----------------------------------------------
\section{Areas of expertise}
\textbf{I have more than 10 years of experience in building and managing HPC software for Research \& Industry in Earth Science.}\\ 
\textbullet{  } Undergraduate \& master's degree in Marine Engineering at Ecole Centrale in France, followed by a PhD in Physical Oceanography at the University of Bordeaux in 2006;  \\
\textbullet{  } CSIRO research scientist for 8 years at the Australian Resources Research Centre;  \\
\textbullet{  } Academic position at the University of Sydney since 2015, work on a diverse range of projects related to Earth evolution; \\
\textbullet{  } \textbf{Software development}: project management, team leadership, technical architecture, scaling \& performance, and training; \\
\textbullet{  } \textbf{Data science}: data analysis, predictive modelling, model scaling and performance optimisation, HPCs \& MPI.

%\vspace{1.5em}
%-----------------------------------------------
%\section{References}
%\cventry{}{Dr Thomas Poulet}{CSIRO Computational Geoscience Group}{\hfill{} \break CSIRO Earth Science \& Resources Engineering}{}{Address: ARRC Building -- PO Box 1130, Bentley WA 6102, Australia. \hfill{} \break  Phone: +61 (0)8 6436 8793, Fax: +61 (0)8 6436 8555,  \hfill{\color{see} \emaillink{thomas.poulet@csiro.au}}}
%\cventry{}{Dr Cedric Griffiths}{CSIRO Predictive Geoscience Group}{\hfill{} \break CSIRO Earth Science \& Resources Engineering}{}{Address: ARRC Building -- PO Box 1130, Bentley WA 6102, Australia. \hfill{} \break  Phone: +61 (0)8 6436 8784, Fax: +61 (0)8 6436 8555,  \hfill{\color{see} \emaillink{cedric.griffiths@csiro.au}}}
%\cventry{}{A/Prof Patrice Rey}{University of Sydney}{\hfill{}\break School of Geosciences - Earthbyte Group}{}{Address: Madsen Building (F09), Rm 437 - The University of Sydney, Australia. \hfill{} \break Phone: +61 (0)2 9351 2067, Fax: +61 (0)2 9351 3644,  \hfill{\color{see} \emaillink{p.rey@usyd.edu.au}}}
%\cventry{}{Dr Marie-Christine Cacas}{Geology \& Geochemistry Division}{\hfill{} \break Institut Fran\c cais du P\'etrole IFP Energies Nouvelles}{}{Address: 1 \& 4, avenue de Bois-Pr\'eau 92852 Rueil-Malmaison Cedex - France. \hfill{} \break Phone: +33 (0)1 47 527 146, Fax: +61 (0)1 47 527 067,  \hfill{\color{see} \emaillink{marie-christine.cacas@ifp.fr}}}
%\cventry{}{Prof Thierry Mulder}{CNRS EVOL}{University of Bordeaux}{}{Address: 1 Avenue des Facult\'es 33405 Talence Cedex France. \hfill{} \break Phone: +33 (0)5 40 008 847, Fax: +33 (0)5 56 840 848,  \hfill{\color{see} \emaillink{t.mulder@epoc.u-bordeaux1.fr}}}

%-----------------------------------------------
\section{Employment history}

%\vspace{0.5em}
\cventry{present--2015}{Academic Position}{The University of Sydney}{\hfill{} \break Lecturer -- School of Geosciences}{EarthByte Group / Geocoastal Group}{}
\cvitem{Research}{\small \vspace{-0.3cm}\begin{itemize}
\item climate change and sediment transport,
\item geodynamics and surface processes feedback mechanisms, 
\item coupled sediment-ocean-wave modelling system at geological scale.
\end{itemize}}
\cvitem{Projects}{\small \vspace{-0.3cm}\begin{itemize}
\item Basin GENESIS Hub -- GEodyNamics and Evolution of SedImentary Systems
\item Discovery Project: Assessing adaptability of Great Barrier Reef to climate change
\item SREI project: Understanding the deep carbon cycle from icehouse to greenhouse climates
\end{itemize}}
\cvitem{Administration}{\small \vspace{-0.3cm}\begin{itemize}
\item Faculty of Science ICT Committee Research Representative
\item IT liaison for the School of Geosciences 
\item Undergrads third year coordinator for the School of Geosciences 
\item Supervision of PhDs  (6), Honours (5), Research Assistants (2) \& Developers (2) 
\end{itemize}}
\cvitem{Teaching}{\small \vspace{-0.3cm} 
\begin{itemize}
\item Marine Science: Coastal processes and systems, Coastal Environments and Processes
\item Computational Geosciences: Global energy \& resources, Geophysical methods
\item Environmental Science:  Environmental Geology,  Environmental Simulation Modelling 
\item Data Science: Analysing and plotting data with R and Python (Strategic Education Grant)
\item Some of my teaching materials can be found here:  \link[$@$GeosLearn]{https://geoslearn.github.io}
\end{itemize}}
\cventry{2015--2012}{Senior Research Scientist}{CSIRO Earth Sciences \& Resources Engineering}{\hfill{}\break Earth Sciences Centre}{Computational Geoscience group, Technical Algorithms team}{Software architect - Project Leader.}
\cvitem{description}{\small \vspace{-0.3cm}\begin{itemize}
\item Project leader on offshore Brazil pre-salt carbonate stratigraphic forward modelling for Petrobras (2.4M\$ over 4 years).
\item Project initiation around Stratigraphic \& Geomorphic forward modelling with Chevron (1.6M\$ over 3 years).
\item Project initiator and software architect of an Advance Earth Dynamic Coalescence Framework using ESMF standard (Ocean Circulation Model - Underworld - Swan 3rd Generation Wave Model - CSIRO Surface Process Model).
\item Project leader for HPC Geosciences project - Integrated Stratigraphy/Seismic Probing/Data Assimilation (National Computational Merit Allocation Scheme - iVec Geosciences Share).
\item Feedback in regional-scale geodynamic/geomorphological systems - oceanic/stratigraphic systems.
\end{itemize}}
\cventry{2012--2011}{Senior Research Scientist}{CSIRO Earth Science \& Resources Engineering}{\hfill{}\break Australian Resources Research Centre}{Predictive Geoscience group}{Software architect - Project Leader.}
\cvitem{description}{\small \vspace{-0.3cm}\begin{itemize}
\item LECO$\Delta$E (SPM) Conception \& Development under Science Innovation Fund of CSIRO Petroleum and Geothermal Portfolio (110k\$ per year for 2 years). 
\item Project leader for HPC Geosciences projects (National Computational Merit Allocation Scheme - iVec Geosciences Share).
\item Feedback in regional-scale geodynamic/geomorphological systems.
\item Development of an inter-divisional and inter-organisational research theme (CSIRO, Sydney University \& Monash University).
\end{itemize}}
\cventry{2011--2008}{Research Scientist}{CSIRO Earth Science \& Resources Engineering}{\hfill{}\break Australian Resources Research Centre}{Predictive Geoscience group}{Computational Geology.}
\cvitem{description}{\small  \vspace{-0.3cm}\begin{itemize}
\item High Performance Computing (HPC) for surface processes models and visualisation methods. 
\item Landscape evolution models and climate/tectonic controls on sedimentary and morphological evolutions of fan/catchment systems. \item Quantitative estimations of sea-level and halokinetics variations on mini-basin filling. \end{itemize}}
\cventry{2008--2007}{Post-Doctoral Research Fellow}{CSIRO Centre for Petroleum Resources}{\hfill{}\break Australian Resources Research Centre}{Predictive Geoscience group}{Predictive Geology.}
\cvitem{description}{\small  \vspace{-0.3cm}\begin{itemize} \item New capabilities in the Stratigraphic Forward Modelling code (SedSim) (Aeolian module based on Cellular Automata paradigm). \item Methodology to simulate climate-change impact on seabed.  \item Mathematical modelling of compaction and diagenesis in sedimentary basins using predictor/corrector implicit finite-difference method.
\end{itemize}}
\cventry{2006--2003}{Doctoral Research Fellow}{French National Research \& Technology funding}{\hfill{} \break University of Bordeaux and Institut Fran\c cais du P\'etrole}{France}{Computational Hydrodynamic.}
\cvitem{description}{\small \vspace{-0.3cm}\begin{itemize} \item Funding highly competitive from the French National Research and Technology Association (CIFRE scholarship).  \item Conception \& development of gravity flows model based on Cellular Automata paradigm.  \item CATS Joint Industrial Project (Cellular Automata Turbiditic System) with sponsorship from Oil Companies.\end{itemize}}
%\cventry{July 2004 -- \\ July 2006}{Geological mapping}{UMR CNRS 5805 EPOC}{}{}{Geology.}
%\cvitem{description}{\small \vspace{-0.3cm}\begin{itemize} \item Participation to 3 missions to characterise the turbidite lobe deposits of the Lauzanier area (SE Alps, France).\end{itemize}}
%\cventry{January 2003 --\\ September 2003}{Research Internship}{LNHE Hydraulic and Environment National Laboratory}{\hfill{}\break EDF Research Centre}{}{Hydraulic Research.}
%\cvitem{description}{\small \vspace{-0.3cm}\begin{itemize} \item The aim of this internship was to characterise experimentally and numerically the impact of waves hydrodynamics stresses on offshore wind pump piles.\end{itemize}} 
\vspace{-1.5em}
%-----------------------------------------------
\section{Education}
%\vspace{0.5em}
\cventry{2017--2018}{Graduate Certificate in Educational Studies (Higher Education)}{\hfill{} \break The University of Sydney}{}{}{}
\cvitem{description}{\small \vspace{-0.3cm}\begin{itemize}\item Postgraduate award course from Sydney School of Education and Social Work. It is an advanced program of study building on candidates' university teaching experience.\end{itemize}} 
\cventry{2007--2008}{CSIRO Postdoctoral Fellow (competitive)}{\hfill{} \break Australian Resources Research Centre -- Centre for Petroleum Research (Perth)}{}{}{}
\cvitem{title}{\emph{Australian Shelf sediment transport responses to climate change-driven ocean perturbations}}
\cvitem{advisors}{Dr C. Dyt and Dr F. Li}
\cvitem{description}{\small \vspace{-0.3cm}\begin{itemize}
\item Predictive assessment of how climate change influences long-term regional seabed responses for CSIRO Wealth from Ocean National Flagship. 
\item Method that could be applied worldwide to assess the impact of climate change-driven ocean perturbations on the seabed. 
\item This research emphasised the importance of quantitative seabed evolution prediction for the sustainable management of coastal/offshore resources and infrastructure, and indicates the most valuable locations for long-term sediment monitoring activity.\end{itemize}} 
\cventry{2003--2006}{Ph.D. in Computational Geology (Hons)}{\hfill{} \break University of Bordeaux \& Institut Fran\c cais du P\'etrole IFP (France)}{}{}{(defended on the 20$^{th}$ of November 2006)}
\cvitem{title}{\emph{Sedimentary filling: Modelling of sub-marine canyons and meandering channels using a genetic approach}}
\cvitem{advisors}{Pr T. Mulder and Dr M.C. Cacas}
\cvitem{description}{\small \vspace{-0.3cm}\begin{itemize}
\item Novel approach to simulate formation and filling of sub-marine canyons and meandering channels by gravity flows (turbidity \& debris flows) which has been patented. \item Algorithm to simulate gravity processes at both reservoir scale and geological time scale. \item Computational technics used: classical approaches based on simplified Navier-Stokes  equations, lattice boltzmann methods for fluid flows and cellular automata paradigm.\end{itemize}}
\cventry{2003--2006}{Petroleum Sciences \& Technics Diploma}{\hfill{} \break ENSPM IFP School (France)}{}{}{}
\cvitem{description}{\small \vspace{-0.3cm}\begin{itemize}\item This formation accessible for Ph.D. students working in IFP provides general knowledge in petroleum studies.  I attended several courses in energy and petroleum production as well as petroleum economics and management.\end{itemize}} 
\cventry{2002--2003}{M.Sc. (research) in Physical Oceanography (Hons)}{\hfill{} \break University of Aix-Marseille II}{}{}{}
\cvitem{description}{\small \vspace{-0.3cm}\begin{itemize}\item I made this one-year degree required before entering doctoral studies in France concomitantly with my last year in engineer school. This formation provides solid knowledge in Oceanography \& Coastal numerical modelling, particularly in: marine/environmental science, coastal physic, turbulence and remote detection.\end{itemize}} 
\cventry{2000--2003}{M.Eng. in Marine Engineering (B.Sc. (Hons) \& M.Eng.)}{\hfill{} \break Ecole Centrale}{}{}{}
\cvitem{description}{\small \vspace{-0.3cm}\begin{itemize}\item This multi-disciplinary formation accessible after selection provides solid knowledge in Mathematics and Physics during the first 2 years and then proposes a specialisation in the last year. I specialised in Marine Engineering, particularly in: numerical implementation \& algorithms, computational physics, deep offshore engineering, ocean hydrodynamic, experimental hydrodynamic, structures dynamic, wave theory and soil engineering.\end{itemize}} 
\vspace{-1.5em}
%-----------------------------------------------
\section{Honours \& Awards}
\cventry{2017}{\small Scientific Mobility Fund}{}{\small one month visitor scientist as part of the Academia Agreement between Statoil and University of Bergen (Norway)}{}{}
\cventry{2016}{\small Scientific Mobility Fund}{}{\small one month visitor scientist as part of the Academia Agreement between Statoil and University of Bergen (Norway)}{}{}
%\cventry{2012}{ Honorary Associate}{\small School of Geosciences}{\hfill{} \break \small The University of Sydney}{}{}
\cventry{2003--2006}{\small IFPEN top-up scholarship}{ \small Convention industrielle de formation par la recherche (CIFRE)}{}{\hfill{} \break \small Industry Research grant}{}{}
\cventry{2003--2006}{ ANRT Scholarship}{\small Association Nationale Recherche et Technologie}{\hfill{} \break \small French Ministry of Research funding}{}{}
\cventry{1997--1999}{ Academic Excellence Scholarship}{}{\small R\'egion Languedoc-Roussillon}{}{}
%\vspace{1.5em}
%-----------------------------------------------
%\section{Responsibilities \& International Projects Involvement}
%\cvitem{\textbf{MARGO}}{Member of the Marine Geoscience Australia experts group. \hfill{} \break \textit{Marine Geoscience Australia}  \hfill \small  \color{see}  \link[margo]{http://www.margo.org.au}}
%\cvitem{\textbf{WA:ERA}}{Member of the WA:ERA for Computational Geology. \hfill{} \break \textit{West Australian Energy Research Alliance}  \hfill \small  \color{see}  \emaillink{www.waera.com.au}}
%\cvitem{\textbf{Research\\programs}}{Member of the \textbf{GdR Marges}. \hfill \small  \color{see}  \emaillink{gdrmarges.free.fr}}
%\cvitem{\textbf{Research\\programs}}{Member of the \textbf{EarthByte Group}. \hfill \small  \color{see}{ \emaillink{www.earthbyte.org}}\break \normalsize \color{black}Member of the \textbf{GdR Marges}. \hfill \small  \color{see}  \emaillink{gdrmarges.free.fr}}
\section{Research grants and major consultancies}
%\vspace{0.5em}
{\small As an Australian Commonwealth-funded organisation employee I was not allow to apply to the Australian Research Council funding schemes as chief investigator from 2007 to 2014. Instead, I have access to competitive grants funded by the CSIRO, and have established successful collaborations and consultancies with the industry and various organisations.}
\small
\begin{tabular}{p{1.9cm}p{7.55cm}p{1.8cm}p{4.5cm}}
\centering \textbf{Year} & \textbf{Short title -- \emph{Role$^*$}} & \centering \textbf{Amount} & \textbf{Funding institution}\\ \\
 \hline 

2017 -- 2018 & Understanding the deep carbon cycle from icehouse to greenhouse climates -- \emph{CI} & \centering \$128k & USYD \\ \\

2016 -- 2017 & Open Learning Education Grant -- \emph{PI} & \centering \$84k & USYD \\ \\

2015 -- 2016 & Computational Geomorphology Supercomputer Allocation Grant -- \emph{CI} & \centering 1M CPU hours & NCMAS \\ \\

2015 -- 2016 & Jupyter-based Ocean Data Resources Allocation Grant -- \emph{CI} & \centering 0.5M CPU hours & NeCTAR Research Cloud \\ \\


2014 -- 2019 & Basin GENESIS Hub (GEodyNamics and Evolution of SedImentary Systems, led by Sydney Uni.) -- \emph{CI} & \centering \$5.4M & Australian Research Council, Statoil, Chevron, OilSearch \\ \\

%2014 -- 2018 & Pre-Salt Carbonates Modelling -- \emph{CI} & \centering \$2.1M & Petrobras \\ \\
%2014 -- 2017 & Coupled Ocean Dynamics \& Surface Processes -- \emph{CI} & \centering \$1.4M & Chevron \\ \\
2013 -- 2014 & Targeting Channel Iron Deposits in the Pilbara -- \emph{PI} & \centering \$130k & MRIWA \& BC Iron Ltd. \\ \\
2012 -- 2013 & Modelling of Channel Iron deposits formation -- \emph{CI}  & \centering \$68k & CSIRO Mineral Flagship\\ \\
2012 -- 2013 & Stratigraphic modelling for South West Collie Hub -- \emph{PI} & \centering \$610k & ANLEC R\&D\\ \\
2011 -- 2012 & Interactions between Continental rifting \& Surface Processes -- \emph{CI} & \centering \$111k & CSIRO Innov. Science Fund\\ \\
2010 -- 2011 & Stratigraphic modelling of Gippsland basin -- \emph{PI} & \centering \$800k & Victoria DPI\\ \\
2009 -- 2011 &  Fluvial \& Aeolian stratigraphic forward modelling of Unayzah formation -- \emph{PI} & \centering \$1.8M & Saudi Aramco \\ \\
2009 -- 2011 &  Controls on confined mini basins filling by eustatic and halokinetic mechanisms -- \emph{CI} & \centering \$110k & ConocoPhillips \\ \\
2009 -- 2010 & Carbonate reservoir facies prediction in the Sichuan Basin -- \emph{PI} & \centering \$1.4M & PetroChina\\ \\
2007 -- 2010 & Impact of Climate Change on Australian Exclusive Economic Zone -- \emph{CI} & \centering \$2.5M & CSIRO Wealth from Ocean Flagship \& BlueLink \\ \\
\hline 
\multicolumn{4}{l}{\emph{{\footnotesize $^*$ CI: chief investigator, PI: partner investigator}}}
\end{tabular}
%\vspace{1.5em}
%-----------------------------------------------
\section{Teaching \& advisor expertise}
{\small During my PhD years, I acted as Teaching Assistant in a number of courses (including a geological mapping course).  In Australia, I have continued my engagement with teaching by developing and delivering a number of post-graduate short courses. I have contributed to the co-supervision of a number of postgraduate students, and in 2010, I have published a textbook on marine sedimentary processes. Since 2015, I am a lecturer within the School of Geosciences in the University of Sydney.}
\vspace{0.5em}
\subsection{B.Sc and M.Sc. teaching experience}
\vspace{0.5em}
\cventry{\small Bordeaux Uni}{Geological mapping}{B.Sc.}{}{One-week field mapping in deep marine environments: Bay of Biscay (SW Pyrenees, France \& Spain) - 2005 }{}

\cventry{\small Bordeaux Uni}{Geological mapping}{M.Sc.}{Two-weeks field mapping of submarine lobes deposits}{Seismic-scale outcrops:  Lauzanier area (SE Alps, France) - 2004, 2005, 2006}{}

\cventry{\small China U. Geosc.}{Numerical tools for Stratigraphic Forward Modelling}{M.Sc.}{}{Introduction to stratigraphic forward model (group teaching), Basin exploration using Sedsim numerical software (exercises) (18h) - 2008}{}

\cventry{\small Uni Sydney}{Coastal processes and systems}{M.Sc.}{}{MARS5001 -- \httplink[unit of study link]{sydney.edu.au/courses/uos/MARS5001/coastal-processes-and-systems}}{}

\cventry{\small Uni Sydney}{Coastal Environments and Processes}{M.Sc.}{}{GEOS3009 -- \httplink[unit of study link]{www.geosci.usyd.edu.au/units_of_study/us_geos3009.shtml}}{}

\cventry{\small  Uni Sydney}{Global energy and resources}{M.Sc.}{}{GEOS3102 -- \httplink[unit of study link]{www.geosci.usyd.edu.au/units_of_study/us_geos3102.shtml}}{}

\cventry{\small Uni Sydney}{Environmental and Sedimentary Geology}{M.Sc.}{}{GEOS3103 -- \httplink[unit of study link]{www.geosci.usyd.edu.au/units_of_study/us_geos3103.shtml}}{}

\cventry{\small Uni Sydney}{Geophysical methods}{M.Sc.}{}{GEOS3104 -- \httplink[unit of study link]{www.geosci.usyd.edu.au/units_of_study/us_geos3104.shtml}}{}

\cventry{\small Uni Sydney}{Environmental Simulation Modelling}{M.Sc.}{}{ENVI5809 -- \httplink[unit of study link]{www.geosci.usyd.edu.au/units_of_study/us_geos3104.shtml}}{}

\vspace{0.5em}
\subsection{Post-graduate short courses}
\vspace{0.5em}
\cventry{\small Honours course}{Tectonic Geomorphology}{}{}{(3 days) - 2015}{}

\cventry{\small CSIRO course}{Using IPython notebook for GIS and input generation}{}{}{(8h) - 2014}{}

\cventry{\small CSIRO course}{Introduction to python programming and data visualisation}{}{}{(12h) - 2013}{}

\cventry{\small Industry course}{Numerical modelling applied to mineral exploration}{}{}{(8h) - 2013}{}

\cventry{\small IGC workshop}{Practical Stratigraphic Forward Modelling}{}{}{IGC Brisbane -- University of Queensland (6h) - 2012}{}

\cventry{\small Industry course}{New visualisation tools for Sedsim model}{}{}{(8h) - 2011}{}

\cventry{\small IAMG workshop}{Using Sedsim for Basin exploration}{}{}{IAMG San Francisco -- Stanford (12h) - 2009}{}

\cventry{IFP consortium}{Cellular Automata models for Marine Geology}{}{}{(4h) - 2005}{}
\vspace{0.5em}
\subsection{M.Sc. advising}
\vspace{0.5em}

\cventry{2015}{Luke Hardiman}{\small University of Sydney (50\%)}{}{\small Honours Project: Landscape dynamic in pull apart basins}{}

\cventry{2015}{Jodie Pall}{\small University of Sydney (50\%)}{}{\small M.Sc Project: Mantle convection and Australian landscape evolution}{}

\cventry{2015}{Samantha Ross}{\small University of Sydney (100\%)}{}{\small M.Sc Project: Post-tectonic landscape recovery}{}

\cventry{2015}{Igor Gomes}{\small University of Rio de Janeiro (100\%)}{}{\small M.Sc Project: Climatic forcing on mountain belt dynamic}{}

\cventry{2014}{Olivia Jobin}{\small University of New Hampshire (30\%)}{}{\small Honours Project: Prediction of tar balls and asphaltite evolution onto Australian beaches in the Great Australian Bight region}{}

\cventry{2013}{Marlene Woligroski}{\small University of Western Australia (30\%)}{}{\small Honours Project: Coupled stratigraphic-seismic modelling of shelf margin depositional sequences}{}

\cventry{2011}{Xiu Huang}{\small China University of Geosciences (20\%)}{}{\small Numerical forward modelling of 'fluxoturbidite' flume experiments using Sedsim model}{}
\vspace{0.5em}
\subsection{Ph.D. advising}
\vspace{0.5em}

\cventry{present--2017}{Megan Holdt}{\small University of Sydney (30\%)}{}{Dynamic Earth models, landscape dynamics, and basin evolution in Eastern Gondwanaland and Zealandia}{}

\cventry{present--2016}{Omer Bodur}{\small University of Sydney (30\%)}{}{\small Mantle Flow and Continental Lithospheric Margins}{}

\cventry{present--2016}{Rhiannon Garrett}{\small University of Sydney (50\%)}{}{\small Numerical modelling of the interplay between tectonic and surface processes in Papua New Guinea}{}

\cventry{present--2015}{Xuesong Ding}{\small University of Sydney (50\%)}{}{\small Dynamic Earth models for frontier hydrocarbon exploration: Stratigraphy and dynamic topography of the North West Shelf}{}

\cventry{2015--2013}{Shailesh Vaidya}{\small University of Western Sydney (40\%)}{}{\small Numerical modelling and theories to predict the extent and rate of  scours below offshore pipelines in calcareous sediments. OCE Postgraduate PhD Scholarship}{}

\cventry{present--2012}{Luke Mondy}{\small University of Sydney (30\%)}{}{\small Continental rifting and the formation of plate margins: insights from numerical modelling and application to the evolution of the North West Shelf, Australia. SIEF John Stocker Postgraduate Scholarship}{}

\cventry{2011--2014}{Valeria Bianchi}{\small Padova University (50\%)}{}{\small Fluvial valley fills accumulated beyond control of relative sea-level changes}{}
\vspace{0.5em}
\subsection{Postdoc advising}
\vspace{0.5em}
\cventry{present--2014}{Valeria Bianchi}{\small University of Queensland (30\%)}{}{\small Surat basin sedimentary modelling}{}

\section{Outreach}
%\vspace{1.0em}
\cvitem{2014}{\httplink[Article on surface processes modelling for iron ore exploration in {Australian Resources \& Investment}, v. 8 (2), p. 72--73]{www.australianresourcesandinvestment.com.au}}
\cvitem{2013}{\httplink[Research highlight in {CSIRO News blog}]{csironewsblog.com/2013/08/21/back-to-the-future-to-find-hidden-riches/}}
\cvitem{}{\httplink[Article on the potential of numerical modelling for Iron ore exploration in {ABC News}]{www.abc.net.au/news/2013-08-19/csiro-unveils-new-iron-ore-detecting-tool/4897198}}
\cvitem{}{Live interview with the Australian Broadcasting Corporation (ABC radio)}
\cvitem{2009}{\httplink[Article on the impact of climate change on Australian seabed  in {Science Daily}]{www.sciencedaily.com/releases/2009/10/091019123111.htm}}

%-----------------------------------------------
\section{Professional invitations, contributions \& affiliations}

\cvitem{Keynote Speaker}{ \small \textit{3D stratigraphic and geomorphic modelling from source to sink}. Symposium 13.7 Modelling sedimentary systems, 34th IGC, Brisbane, 2012.}

\cvitem{Invited Speaker}{ \small \textit{GEMOC seminar}, Macquarie University, Sydney, March 2013.}

\cvitem{}{ \small \textit{GeoPRISMS/CSDMS Geodynamics Focus Research group workshop}, AGU Fall meeting, San Francisco, December 2013.}

\cvitem{}{ \small \textit{SedSim model challenges and developments apply to Ore Deposits R\&D}. Les Rencontres Scientifiques de l�IFP, Paris, 2010.}

\cvitem{}{ \small \textit{Climate change impact on seabed sediment transport all around Australian EEZ}. Australia and New Zealand Industrial and Applied Mathematics meeting (ANZIAM), University of Western Australia, 2008.}

\cvitem{}{ \small \textit{Cellular Automata paradigm applied to gravity flows modelling}. Exxonmobil Upstream Research seminar, Houston, 2007.}

\cvitem{}{ \small \textit{Sedimentary filling: Modelling of sub-marine canyons and meandering channels using a genetic approach}. DIONISOS JIP Consortium, IFP Paris, 2006.}

\cvitem{Invited Author}{ \small \textit{Simulation of the interactions between gravity processes and contour currents on the Algarve Margin (South Portugal) using the stratigraphic forward-model SedSim}. Sedimentary Geology (229, 3), 2010 -- Lobes in deep-sea turbidite systems.}

\cvitem{}{ \small \textit{A Turbidity Currents Model to Simulate Impact of Basin-Scale Forcing Parameters}. SEPM Society for Sedimentary Geology special publication 92, 2009 -- External Controls on Deep-Water Depositional Systems.}

\cvitem{Conference \\ Organisation}{ \small  \textbf{ICS2016}  -- International Coastal Symposium \hfill{} \break  \textit{Member of the scientific \& organising committee}}  
\cvitem{}{ \small \textbf{ASF} Conference (Association des S\'edimentologistes Fran\c cais) -- Bordeaux, 2003  \hfill{} \break  \textit{Organisation Committee}}

\cvitem{Projects \\ Involvement}{\textbf{MARGO} Member of the Marine Geoscience Australia experts group. \hfill{} \break \textit{Marine Geoscience Australia} }
\cvitem{}{\textbf{WA:ERA} Member of the WA:ERA for Computational Geology. \hfill{} \break \textit{West Australian Energy Research Alliance} }

\cvitem{}{Member of the \textbf{GdR Marges}.}

%-----------------------------------------------
\section{Computer skills}

\cvcomputer{OS}{OS X, Linux, Unix, Windows}{visualisation}{VTK, NetCDF, HDF5}
\cvcomputer{programming}{Python, XML, C, C++, Fortran}{typography}{\LaTeX, Microsoft Office, iWork}
\cvcomputer{HPC}{MPI, OpenMP}{}{}

%-----------------------------------------------
\section{Organisations membership}

\cvitem{since 2000}{\small Member of \textbf{Centrale Alumni}}
\cvitem{since 2003}{\small Member of \textbf{IFP School Alumni}}
\cvitem{2003-2006}{\small Member of Society of Petroleum Engineers -- \textbf{SPE}}
\cvitem{since 2010}{\small Member of American Geological Union -- \textbf{AGU}}
\cvitem{since 2010}{\small Member of European Geological Union -- \textbf{EGU}}
\cvitem{since 2012}{\small Member of the French Researchers in Australia Network -- \textbf{FRAN}}
\cvitem{since 2012}{\small Member of the Community Surface Dynamics Modeling System -- \textbf{CSDMS}}

%-----------------------------------------------
\section{Publication history} 

\cvitem{\textbf{Book}}{\small \textbf{Salles T.}, 2010. \textit{Mod\'elisation num\'erique du remplissage s\'edimentaire des canyons et chenaux sous-marins par Approche G\'en\'etique}, Edition Universitaire Europ\'eenne eds, 215 p.}

\cvitem{\textbf{Journal Papers}}{\small  \textbf{Salles T.}, Flament N., Muller D., 2017.  \textit{Influence of mantle flow on the drainage of eastern Australia since the Jurassic Period}. Geochem. Geophys. Geosyst., in Press}

\cvitem{2016 }{\small  \textbf{Salles T.}, 2016. \textit{Badlands: A parallel basin and landscape dynamics model}. SoftwareX, 5, 195-202.}

\cvitem{ }{\small  \textbf{Salles T.}, Hardiman. L.., 2016. \textit{Badlands: an open-source, flexible and parallel framework to study landscape dynamics}. Computers and Geosciences, 91, 77-89.}

\cvitem{2015 }{\small  \textbf{Salles T.}, 2015. \textit{Badlands: a parallel basin and landscape dynamics model}. SoftwareX, under review.}

\cvitem{ }{\small Bianchi V., \textbf{Salles T.}, Ghinassi M., Billi P., Dallanave E., Duclaux G., 2015. \textit{Numerical modelling of tectonically driven river dynamics and deposition in an upland incised valley}. Geomorphology, 241, 353-370.}

\cvitem{ }{\small \textbf{Salles T.}, Duclaux G., 2015. \textit{Combined hillslope diffusion and sediment transport simulation applied to landscape dynamics modelling}. Earth Surface Processes and Landforms, 40(6), 823-839.}

\cvitem{2013}{\small Duclaux G., \textbf{ Salles T.}, Ramanaidou E., 2013. \textit{Alluvial iron deposits exploration using surface processes modelling: A case study in the Hamersley Province (WA).} Iron Ore Conference journal proceeding.}

\cvitem{2012}{\small Miranda J., Eid R., McLean M., O'Brien G., Griffiths C., Dyt C.,  \textbf{Salles T.}, Tingate P., Goldie Divko L., Campi M., 2012. \textit{Gippsland Basin stratigraphic and CO2 migration modelling: workflows for building regional, geological carbon storage (GCS) reservoir models}. EABSiv Publication.}

\cvitem{}{\small Etienne S., Mulder T., Bez M., Desaubliaux G, Kwasniewski A., Parize O., Dujoncquoy E., \textbf{ Salles T.}, 2012. \textit{Multiple scale characterization of sand-rich distal lobe deposit variability: Examples from the Annot Sandstones Formation, Eocene-Oligocene, SE France}. Sedimentary Geology, v. 273, p. 1-18.}

\cvitem{}{\small Huang X., Dyt C., Griffiths C., \textbf{ Salles T.}, 2012. \textit{Numerical forward modelling of fluxoturbidite flume experiments using Sedsim}. Marine and Petroleum Geology, v. 35, p. 190-200.}

\cvitem{2011}{\small \textbf{Salles T.}, Griffiths C., Dyt C.  \textit{Aeolian sediment transport integration in general stratigraphic forward modelling}. Journal of Geological Research, vol. 2011-186062, 12 p.}

\cvitem{}{\small \textbf{Salles T.}, Griffiths C., Dyt C., Li F., 2011.  \textit{Australian Shelf sediment transport responses to climate change-driven ocean perturbations}. Marine Geology, vol. 282, p. 268-274.}

\cvitem{2010}{\small \textbf{Salles T.}, March\'es E., Dyt C., Griffiths C., Hanquiez V., Mulder T, 2010.  \textit{Simulation of the interactions between gravity processes and contour currents on the Algarve Margin (South Portugal) using the stratigraphic forward-model SedSim}. Sedimentary Geology, vol. 229, p.  95-109.}

\cvitem{}{\small  Mulder T., Callec Y., Parize O., Joseph P., Schneider J.-L., Robin C., Dujoncquoy E., \textbf{Salles T.}, Allard J., Ferger B. Hanquiez V., March\'es E. Toucanne S.and Zaragosi S, 2010.  \textit{The turbidite lobe deposits of the Lauzanier area (SE Alps, France)}. Sedimentary Geology, vol. 229, p. 160-191.}

\cvitem{2009}{\small Li F., Griffiths C., Dyt C., Weill P., Feng M., \textbf{Salles T.}, Jenkins C., 2009.  \textit{Multigrain seabed sediment transport modelling for the south-west Australian Shelf}. Marine and Freshwater Research, 60(7), p. 774-785.}

\cvitem{2008}{\small \textbf{Salles T.}, Lopez S., Eschard R., Mulder T., Euzen T., Cacas M.C., 2008. \textit {A Turbidity Currents Model to Simulate Impact of Basin-Scale Forcing Parameters}. in Kneller, B., Martinsen, O.J., and McCaffrey, B., eds., External Controls on Deep-Water Depositional Systems: SEPM Special Publication 92, p. 363-384.}

\cvitem{}{\small Li F., Griffiths C. M., \textbf{Salles T.}, Dyt C. P., Feng M., Jenkins C., 2008. \textit{Climate change impact on NW shelf seabed evolution and its implication on offshore pipeline design}. Refereed paper, APPEA Journal.}

\cvitem{}{\small \textbf{Salles T.}, Cacas M.C., Mulder T., Li F., Griffiths C. M., Dyt C. P., 2008. \textit{Sedimentary fill of submarine canyons and channels using a Cellular Automata process-based model}. Refereed paper, APPEA Journal.}

\cvitem{}{\small \textbf{Salles T.}, Mulder T., M. Gaudin, Cacas M.C., Lopez S., Cirac P., 2008. \textit{Simulating the 1999 turbidity current occurred in Capbreton canyon through a Cellular Automata model}. Geomorphology, vol. 97, issues 3-4, p. 516-537.}

\cvitem{}{\small \textbf{Salles T.}, Lopez S., Eschard R., Lerat O., Mulder T., Cacas M.C., 2008. \textit{Turbidity current modelling on geological time-scales}. Marine Geology, vol. 248, issues 3-4, p. 127-150.}

\cvitem{2007}{\small \textbf{Salles T.}, Lopez S., Cacas M.C., Mulder T., 2007. \textit{Cellular Automata Models of Density Currents}. Geomorphology, vol. 88, issues 1-2, p. 1-20.}

\cvitem{\textbf{Patent}}{\small \textbf{Salles T.}, Lopez S., Joseph P., Cacas M.C., 2007. \textit{Use of the stable condition of cellular automata exchanging energy to model sedimentary architectures}, EP1837683.}

\cvitem{\textbf{Technical manuals}}{\small \textbf{Salles T.}, Duclaux G., Mondy L., Bianchi V., 2013. \textit{SGFM: Stratigraphic \& Geomorphic Forward Modelling Framework}. CSIRO Scientific \& Technical Publication.}

\cvitem{2011}{\small \textbf{Salles T.}, Duclaux G., 2011. \textit{Tellus: surface processes modelling software}. CSIRO Scientific \& Technical Publication.}

\cvitem{2009}{\small \textbf{Salles T.}, 2009. \textit{SedSimVisu for Paraview: A user guide}. Software Manual. }


\cvitem{\textbf{Conference abstracts}}{\textbf{T. Salles}, 2017. \textit{Responses of reefs to climatic forcing -- A numerical perspective}, Centre for Coral Reef Studies, USyd.}

\cvitem{}{\small \textbf{T. Salles}, N. Flament, D. Muller, P. Rey, 2017. \textit{Influence of dynamic topography on the evolution of the Australian landscape since the Late Jurassic}, Workshop in High Performance Computing, EAGE IFP Paris.}

\cvitem{}{\small \textbf{T. Salles}, N. Flament, D. Muller, 2017. \textit{150 Million years of landscape evolution of eastern Australian continent}, European Geophysical Union General Assembly Vienna, Austria.}

\cvitem{}{\small X. Ding, \textbf{T. Salles}, N. Flament, P. Rey, 2017. \textit{Influence of dynamic topography on landscape evolution and passive continental margin stratigraphy}, European Geophysical Union General Assembly Vienna, Austria.}

\cvitem{2016}{\small \textbf{T. Salles}, 2016. \textit{Regional to continental scale model of Earth surface evolution}, Seminar - Granular Forum, USyd.}

\cvitem{}{\small \textbf{T. Salles}, 2016. \textit{Update on Badlands development \& applications}, Basin Genesis Hub meeting, Melbourne.}

\cvitem{}{\small X. Ding, \textbf{T. Salles}, N. Flament, P. Rey, 2016. \textit{Impact of dynamic topography on stratigraphic evolution}, Basin Genesis Hub meeting, Melbourne.}

\cvitem{}{\small \textbf{T. Salles}, N. Flament, D. Muller, 2016. \textit{Influence of dynamic topography on the evolution of the Australian landscape since the Late Jurassic}, Australian Earth Science Convention.}

\cvitem{}{\small V. Bianchi, T. Smith, \textbf{T. Salles}, J. Esterle, 2016. \textit{Coupling dynamic topography with Stratigraphic Forward Modelling: case study Springbok Sandstone}, Australian Earth Science Convention.}

\cvitem{}{\small D. Muller, N. Flament, \textbf{T. Salles}, K. Matthews,  M. Gurnis, 2016. \textit{The eastern Australian record of continental travel, dynamic topography and landscape evolution since Pangaea breakup}, Australian Earth Science Convention. }

\cvitem{}{\small \textbf{T. Salles}, 2016. \textit{Continental-Scale Landscape Dynamic}, University of Bergen, Norway }

\cvitem{}{\small \textbf{T. Salles}, N. Flament, D. Muller, 2016. \textit{Influence of dynamic topography on the evolution of the Australian landscape since the Late Jurassic}, American Geophysical Union.}

\cvitem{}{\small X. Ding, \textbf{T. Salles}, N. Flament, P. Rey, 2016. \textit{Modeling passive margins stratigraphy from the interplay between sea-level change, thermal subsidence, precipitation and dynamic topography}, American Geophysical Union.}

\cvitem{}{\small S. Moron, S. Gallagher, L. Moresi, \textbf{T. Salles}, P. Rey, T. Payenberg, 2016. \textit{Investigating the effect of plate-mantle interaction in basin creation and associated drainage systems: insights from the North West Shelf of Australia}, American Geophysical Union.}

\cvitem{2015}{\small M�ller D,  \textbf{T. Salles}, N. Flament, M. Gurnis, 2015. \textit{Continental inter-superswell travel and landscape evolution} -- GeoBerlin, Gemany, 2015.}

\cvitem{}{\small Bianchi V., \textbf{T. Salles}, 2015. \textit{Stratigraphic Forward Modelling on reservoir geometry of the Springbok Formation (Late Jurassic)} -- Bowen Basin Symposium and Beyond October 2015 (Brisbane)}

\cvitem{}{\small M�ller D, N. Flament, K. Matthews, S. Williams, M. Gurnis, \textbf{T. Salles}, 2015. \textit{How Australian Plate Interaction With Subducting Slabs and the South Pacific Superswell Drove Multi-Phase Uplift and Paleogeography in Eastern Australia} -- ICE-SEG AAPG, Melbourne, Australia, September 2015.}


\cvitem{}{\small Bianchi V., \textbf{T. Salles}, J. Esterle, 2015. \textit{Stratigraphic Forward Modelling for investigating hidden reservoir geometries and connectivity: Springbok Formation } -- ICE-SEG AAPG, Melbourne, Australia, September 2015.}

\cvitem{}{\small Flament N., \textbf{T. Salles}, R.D. M�ller, M. Gurnis, 2015. \textit{Influence of subduction history and surface processes on continental-scale topography} -- XIV International Workshop on Modelling of Mantle and Lithosphere Dynamics, Ausgust 2015 (France)}

\cvitem{}{\small Bianchi V., J. Esterle, \textbf{T. Salles}, 2015. \textit{Stratigraphic Forward Modelling on reservoir geometry of the Springbok Formation (Middle Jurassic): prediction on the unexplored depocenter of Surat Basin (QLD, Australia) } -- IAS June 2015 Krakow (Poland)}

\cvitem{}{\small Bianchi V., \textbf{T. Salles}, 2015. \textit{Hidden geometry in Surat Basin depocenter: Springbok Formation} -- AAPG Geosciences Technology Workshop (GTW) entitled: Opportunities and Advancements in Coal Bed Methane in the Asia Pacific - January 2015 (Brisbane)}

\cvitem{2014}{\small M�ller D, P. Rey, L. Moresi, L. Mondy, G. Duclaux, \textbf{T. Salles}, T. Rawling, C. Elders, 2014. \textit{Next generation modelling of rift basins and continental margins} -- Australian Earth Science Convention, Newcastle, Australia, July 2014.}

\cvitem{}{\small Mondy L., P. Rey, G. Duclaux, \textbf{T. Salles}, L. Moresi, 2014. \textit{A digital workbench for understanding the stratigraphic evolution of rift basins and continental margins} -- Australian Earth Science Convention, Newcastle, Australia, July 2014.}

\cvitem{2013}{\small Upton, P., G. Duclaux,  D. Craw, \textbf{T. Salles}, 2013. \textit{Reconstructing the formation and in-filling of Lake Manuherikia, Otago: Linking geodynamics and surface processes} -- AGU Fall meeting 2013, San Francisco, USA, December 2013.}

\cvitem{}{\small Mondy L., Duclaux G., \textbf{ Salles T.}, Thomas C., Rey P.,  2013. \textit{Modelling stratigraphic and surface dynamics processes on a coupled thermo-mechanical lithospheric model: an example in oblique continental rifting}. IAG Paris.}

\cvitem{}{\small Bianchi, V., \textbf{T. Salles}, G. Duclaux, M. Ghinassi, 2013. \textit{Reconstruction of a syn-depositional cross-valley faulting through numerical modelling: the Plio-Pleistocene Ambra paleovalley (North Apennines, Italy)} -- GeoSed, Roma, Italy, September 2013.}

\cvitem{}{\small Bianchi V., \textbf{ Salles T.}, Duclaux G.,, Ghinassi M.,  2013. \textit{Evolution of syn-depositional cross-valley faulting: numerical reconstruction of the Plio-Pleistocene Ambra paleovalley (Northern Apennines, Italy)}. IAS, Manchester, UK, September 2013..}

\cvitem{}{\small Mondy L., Duclaux G., \textbf{ Salles T.}, Thomas C., Rey P.,  2013. \textit{Stratigraphic Modelling of Continental Rifting}. EGU 2013, Vienna, Austria, May 2013.}

\cvitem{}{\small Bourget J., \textbf{ Salles T.}, Ainsworth B., Duclaux G., 2013. \textit{Quantifying the importance of sediment supply, global eustasy and fault-induced accommodation in controlling delta architecture, shelf-margin growth and deep-water sediment transfer: insights from stratigraphic-forward modelling in northern Australia.}. AAPG Annual Convention Denver.}

\cvitem{}{\small Duclaux G., \textbf{ Salles T.}, 2013. \textit{Four-Dimensional Detrital Iron Ore Deposit Exploration Using Surface Processes Modelling - Hamersley Province Example.}. Iron Ore conference Perth.}

\cvitem{2012}{\small \textbf{Salles T.}, Duclaux G., 2012. \textit{3D stratigraphic and geomorphic modelling from source to sink}, 34th IGC, Brisbane.}

\cvitem{}{ \small Duclaux G., \textbf{Salles T.}, 2012. \textit{Placer deposits: where should we look next? Surface Processes Modelling applied to mineral exploration}, 34th IGC, Brisbane.}

\cvitem{}{ \small \textbf{Salles T.}, Duclaux G., 2012. \textit{Next generation of 3D stratigraphic and geomorphic modelling}, EGU, Vienna.}

\cvitem{}{ \small \textbf{Salles T.}, Duclaux G., Mansour J., Rey P., Moresi L., 2012. \textit{Testing basin-scale evolution concepts on a supercomputer at realistic tectonic and sedimentary rates}, EGU, Vienna.}

\cvitem{}{ \small Miranda J., Eid R., McLean M., Griffiths C., Dyt C.,  \textbf{Salles T.}, O'Brien G., Tingate P., Divko L. G., Campi M., 2012. \textit{Gippsland basin stratigraphic and CO2 migration modelling: workflows for building regional, geological carbon storage (GCS) reservoir models}, Eastern Australian Basins Symposium IV, Brisbane.}

\cvitem{}{ \small Griffiths C., Dyt C., Huang X., \textbf{Salles T.}, 2012. \textit{Multiscalar Forward modelling of carbonate heterogeneity}, AAPG/SPE/SEG Hedberg Conference France.}

\cvitem{2011}{ \small Duclaux, G.,  \textbf{Salles T.}, Rey P., 2011. \textit{Numerical modelling of tectonic and surface processes: insights into continental rifting and the formation of passive margins}, GSA Annual Meeting 2011, Minneapolis, USA, October 2011.}

\cvitem{}{ \small \textbf{Salles T.}, Duclaux G., 2011. \textit{Tellus: A new HPC model based on particle-in-cell technique to investigate stratigraphy evolution}, GSA Annual Meeting 2011, Minneapolis, USA, October 2011. }

\cvitem{}{ \small  Griffiths C., Dyt C., Huang X., \textbf{Salles T.}, Corbel S., 2011. \textit{Predicting Rock Properties away from data in Aquifer-hosted Geothermal Projects}, West Australian Geothermal Energy Symposium, Perth.}

\cvitem{2010}{ \small \textbf{Salles T.}, Duclaux G., 2010. \textit{SedSim model challenges and developments apply to Ore Deposits R\&D}, IFP Paris, 2010.}

\cvitem{}{ \small \textbf{ Salles T.}, McGee D., Griffiths C., Ryer M. S., 2010. \textit{SedSim modelling of controls on confined mini basin fill by eustatic and halokinetic mechanisms (Gulf of Mexico)}, symposium ASF, RST 2010.}

\cvitem{2009}{ \small \textbf{Salles T.}, Griffiths C., Dyt C., 2009. \textit{Aeolian sediment transport integration in general stratigraphic forward modelling}. IAMG Conference, Stanford.}

\cvitem{}{ \small Griffiths C., Dyt C., \textbf{Salles T.}, 2009. \textit{The 2009 Status of SedSim modelling and future plans}. IAMG Conference, Stanford.}

\cvitem{}{ \small Griffiths C., \textbf{Salles T.}, Li F., Dyt C., 2009. \textit{Using SedSim to predict the response of coastal sediments to climate change in Australia}. Climate Change: The environmental and socio-economic response in the southern Baltic region. Poland.}

\cvitem{}{ \small \textbf{ Salles T.}, 2009. \textit{Climate change impact on seabed sediment transport in the Australian exclusive economic zone}. Society of Underwater Techonology Annual Conference, Perth.}

\cvitem{2008}{ \small \textbf{ Salles T.}, Li F., Griffiths C. M., Dyt C. P., Feng M., Jenkins C., 2008. \textit{Climate change impact on seabed sediment transport in the Australian Exclusive Economic Zone}. Coast to Coast Conference, Darwin.}

\cvitem{}{ \small \textbf{ Salles T.}, Li F., Griffiths C. M., 2008. \textit{Climate change impact on seabed sediment transport all around Australian}. ANZIAM Conference Perth.}

\cvitem{}{ \small Eschard R., Teles V.,  \textbf{ Salles T.}, Lopez S., 2008. \textit{Simulation of the stratigraphic architecture of the PAB turbiditic channels: the additional value of a process-based numerical approach with the CATS model}. AAPG, San Antonio.}

\cvitem{}{ \small Teles V., Eschard R., Lopez S.,  \textbf{ Salles T.}, 2008. \textit{A Process-based Cellular Automata Model for Turbiditic Reservoirs (CATS) Applied to Complex Turbiditic Systems}. AAPG, San Antonio.}

\cvitem{2007}{ \small \textbf{ Salles T.}, Lopez S., Cacas M.C., Eschard R., Euzen T. , Teles V., Mulder, T., 2007. \textit{Cellular Automata Modelling of Turbidity Currents Deposits}. AAPG Annual Convention Exhibition, Long Beach.}

\cvitem{}{ \small \textbf{ Salles T.}, Mulder T., Gaudin M., Lopez S., Cacas M.C., Cirac P., 2007. \textit{Simulating the 1999 turbidity current in Capbreton canyon (French Atlantic Coast) using a Cellular Automata model}. EGU Vienna.}

\cvitem{2006}{ \small \textbf{ Salles T.}, Lopez S., Euzen T., Eschard R., 2006. \textit{Integrating basin-scale forcing parameters in density currents process-based modelling}. External Controls on Deep Water Depositional Systems; Climate, Sea-Level and Sediment Flux, London. }

\cvitem{}{ \small Granjeon D., Wolf S., Lopez S., \textbf{ Salles T.}, 2006. \textit{3D Stratigraphic Modelling in Complex Tectonics Area at an Exploration to Reservoir Scale}. AAPG Annual Convention, Houston. }

\cvitem{}{ \small \textbf{ Salles T.}, Lopez S., Cacas M.C., Granjeon D., Mulder T., 2006. \textit{Geological Modelling Using Cellular Automata}. AAPG Annual Convention, Houston. }

\cvitem{2005}{ \small \textbf{ Salles T.}, Cacas M.C., Mulder T., 2005. \textit{Mod\'elisation num\'erique du remplissage s\'edimentaire des canyons et chenaux sous-marins par Approche G\'en\'etique}. Congr\'es ASF (Giens), France.}

\cvitem{}{ \small Mulder T., Callec Y., Joseph P., Robin C., Schneider J.L., \textbf{Salles T.}, Allard J., Ferger B., Bonnel C., Cremer M., Ducassou E., Dujoncquoy E., Gaudin M., Hanquiez V., March\'es E., Parize O., Toucanne S., Zaragosi S., 2005. \textit{Les d\'epots de lobes turbiditiques du lac du Lauzanier (Ubaye, France)}. Congr\'es ASF (Giens), France. }

\cvitem{}{ \small \textbf{ Salles T.}, Cacas M.C., Mulder T., 2005. \textit{Simulating Turbidity Currents through a Cellular Automata Model}. AAPG International Conference, Paris.}
 
\cvitem{2004}{ \small \textbf{ Salles T.}, Cacas M.C., Mulder T., 2004. \textit{Cellular automata approach for simulated gravity flows}. Modelling of Turbidity Currents and Related Gravity Currents Workshop, Santa Barbara.}

\cvitem{\textbf{Refereed \\ reports}}{\small Duclaux G., \textbf{T. Salles}, E. Ramanaidou, 2014. MRIWA project 433: Targeting Channel Iron Deposits formation in the Pilbara using a forward landscape evolution and clastic sedimentation approach, stage 1 report -- CSIRO publication  (confidential) , Australia, 35 p.}

\cvitem{2013}{\small \textbf{Salles T.}, Duclaux G., Mondy L., Bianchi V., 2013. \textit{SGFM: Stratigraphic \& Geomorphic Forward Modelling Framework}. CSIRO Publication nb. 13-001.}

\cvitem{2012}{\small Griffiths C., Seyedmehdi Z., \textbf{Salles T.}, Dyt C., 2012. \textit{Stratigraphic Forward Modelling for South West  Collie Hub}. CSIRO report for ANLEC.}

\cvitem{}{\small \textbf{Salles T.}, G. Duclaux, 2012. \textit{Surface Processes Modelling Methodology} -- CSIRO publication (internal report), 56 p.}

\cvitem{2011}{ \small Griffiths C., \textbf{Salles T.}, Dyt C., 2011. \textit{Stratigraphic forward modelling of the Gippsland basin from early Campanian (85Ma) to Recent}. CSIRO report for VicDPI.}

\cvitem{2010}{ \small Griffiths C., \textbf{Salles T.}, Dyt C., 2010. \textit{Hawtah Trend - Predicting Unayzah facies distribution using SedSim}. CSIRO Confidential report for Saudi Aramco.}

\cvitem{}{ \small Duclaux G., \textbf{Salles T.}, Hodkinson L., 2010. \textit{Recent developments of a coupled tectonics and surface processes framework}, CSIRO Exploration \& Mining Science Review, Brisbane.}

\cvitem{2009}{ \small \textbf{Salles T.}, Griffiths C., Dyt C., Jenkins C. , Oke P., Feng M., 2009. \textit{Modelled seabed response to possible climate change scenarios over the next 50 years in the Australian NE}. CSIRO Publication nb. 09-001.}

\cvitem{}{ \small \textbf{Salles T.}, 2009. \textit{Aeolian model developed in SedSim}. CSIRO Publication nb. 09-002.}

\cvitem{}{ \small \textbf{Salles T.}, 2009. \textit{Mathematical modelling of compaction and diagenesis in sedimentary basins}. CSIRO Publication nb. 09-003.}

\cvitem{2007}{ \small Li F., \textbf{Salles T.}, Griffiths C. M., Dyt C. P., Jenkins C. , Oke P., Feng M., 2007. \textit{Modelled seabed response to possible climate change scenarios over the next 50 years in the Australian NW Shelf}. CSIRO Publication - report 07-005.}

\cvitem{2006}{ \small \textbf{Salles T.}, 2006. \textit{Mod\'elisation num\'erique du remplissage s\'edimentaire des canyons et chenaux sous-marins par Approche G\'en\'etique}. PhD Thesis n� 3254, 215 p.}
 
\cvitem{2004}{ \small \textbf{Salles T.}, 2004. \textit{Les \'ecoulements gravitaires dans les syst\'emes sous-marins profonds, les formations s\'edimentaires associ\'ees et les mod\'elisations r\'ealis\'ees}, rapport IFP: 58231, 184 p. }

\cvitem{2003}{ \small \textbf{Salles T.}, 2003. \textit{Etude des for�ages hydrodynamiques exerc\'es par la houle sur les piles \'eoliennes offshore par approche num\'erique et exp\'erimentale}, LNHE report, 125 p.}

\cvitem{\textbf{Grant}}{ \small Rey P., Duclaux G., \textbf{Salles T.}, 2011. Continental rifting and the formation of plate margins: insights from numerical modelling and application to the evolution of the North West Shelf, Australia. SIEF John Stocker Postgraduate Scholarship.}

\cvitem{\textbf{Non-refereed \\ reports}}{ \small \textbf{Salles T.}, 2013. \textit{CSIRO Stratigraphic \& Geomorphic Forward Modelling - Promoting impact Science \& Strategic Plan horizon 2020.} CSIRO Strategic Planning}

\cvitem{2009}{ \small \textbf{Salles T.}, Griffiths C., 2009. \textit{SedSim modelling of deepwater basin. Report for Conocco Phillips on mini basin study in Gulf of Mexico area}}

\cvitem{}{\small  Wiki construction \& maintenance for the Predictive Geoscience Group since 2009.}
\cvitem{}{\small Griffiths C., \textbf{ Salles T.}, 2009. \textit{Environmental seabed modelling}. CSIRO Publication.}

%-----------------------------------------------

\end{document}
